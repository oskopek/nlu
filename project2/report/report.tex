\documentclass{article}

% if you need to pass options to natbib, use, e.g.:
\PassOptionsToPackage{square,numbers,comma,sort}{natbib}
\bibliographystyle{plain}
% before loading nips_2017
%
% to avoid loading the natbib package, add option nonatbib:
% \usepackage[nonatbib]{nips_2017}

\usepackage[final]{nips_2017}


\usepackage[utf8]{inputenc} % allow utf-8 input
\usepackage[T1]{fontenc}    % use 8-bit T1 fonts
\usepackage{hyperref}       % hyperlinks
\usepackage{url}            % simple URL typesetting
\usepackage{booktabs}       % professional-quality tables
\usepackage{amsfonts}       % blackboard math symbols
\usepackage{nicefrac}       % compact symbols for 1/2, etc.
\usepackage{microtype}      % microtypography

% Choose a title for your submission
\title{NLU Task 2: Story Cloze Task --- Report}

\author{
  Luk\'{a}\v{s} Jendele$^\ast$\\
  ETH Zurich\\
  \texttt{jendelel@ethz.ch}\\
  %% examples of more authors
  \And
  Ondrej Skopek$^\ast$\\
  ETH Zurich\\
  \texttt{oskopek@ethz.ch}\\
  \And
  Vasily Vitchevsky$^\ast$\\
  ETH Zurich\\
  \texttt{vasilyv@ethz.ch}\\
  \And
  Micheal Wiegner\thanks{All authors contributed equally.}\\
  ETH Zurich\\
  \texttt{wiegnerm@ethz.ch}\\
}

\begin{document}
\maketitle

% We do not requrire you to write an abstract. Still, if you feel like it, please do so.
\begin{abstract}
TODO Abstract
\end{abstract}

Feel free to add more sections but those listed here are strongly recommended.

\section{Introduction}\label{sec:intro}
You can keep this short.
Ideally you introduce the task already in a way that highlights the difficulties your method will tackle.\citep{StoryCloze}

\section{Methodology}\label{sec:methodology}
Your idea.
You can rename this section if you like.
Early on in this section -- but not necessarily first -- make clear what category your method falls into: Is it generative?
Discriminative?
Is there a particular additional data source you want to use?

\section{Model}\label{sec:model}
The math/architecture of your model.
This should formally describe your idea from above.
If you really want to, you can merge the two sections.

\section{Training}\label{sec:training}
What is your objective?
How do you optimize it?

\section{Experiments}\label{sec:experiments}
This \textbf{must} at least include the accuracy of your method on the validation set.

\section{Conclusion}\label{sec:conclusion}
You can keep this short, too.

\bibliography{bibliography}
\end{document}
